\documentclass[12pt]{article}

\usepackage[margin=1in]{geometry}

\begin{document}
\pagenumbering{gobble}
\begin{center}
\bfseries
\LARGE
CMPSC 293S Winter 2019\\
Project Definition\\[2em]
\end{center}

\noindent
\textbf{Group Name}: The Shambles\\
\noindent
\textbf{Group Members}:
\begin{table}[h]
\centering
\begin{tabular}{|l|l|l|l|}
\hline
\textbf{Last Name} & \textbf{First Name} & \textbf{Email} & \textbf{Student ID}\\
\hline
Saquib & Nazmus & nazmus@ucsb.edu & 7275274\\
\hline
Paul & Udit & u\_paul@ucsb.edu & 6610687\\
\hline
Ermakov & Alex & aermakov@ucsb.edu & 8075517\\
\hline
Ramamoorthy & Santha Meena & santhameena@ucsb.edu & 6404842\\
\hline
\end{tabular}
\end{table}

\noindent
\textbf{Project Description}: Proper irrigation is an important fator for growth and development of plants and crop. The intensity of irrigation depends on the amount of water lost from the land. If too little water is used for irrigation compared to the amount lost, plants/crops will not get enough water. On the other hand, if too much water is used it would have a detrimental effect on the plants/crops. Water can be lost in primarily two forms -- evaporation from the surface of the soil and transpiration from aerial parts of plant. Together this is known as \textit{evapotranspiration} (ET). ET depends on a number of parameters such as temperature, soil moisture, solar irradiation, etc. In this project we will try to estimate ET values by studying the SmartFarm data along with data from California Irrigation Management Information System (CIMIS -- has weather stations in different places). The primary objectives of our project are delineated below. More objectives might be added as the project progresses and/or the already stated objectives might be refined.
\begin{enumerate}
	\item \textit{Estimation of ET values from CIMIS data}. CIMIS contains data for some of the parameters required to calculate ET along with their own value of ET. It could be observed how close we can get to the ET value published in CIMIS using their paramter values.
	\item \textit{Estimation of ET values by augmenting local data}. Data from SmartFarm sensors might be plugged in place of some data from CIMIS. Ideally we would want the estimate to improve in this case.
	\item \textit{Estimation of ET values by combining reading from multiple weather stations}. Instead of trying to estimate ET using data from only one weather stations, nearby $k$ weather stations could be observed to get an estimate.
	\item \textit{Estimation of ET values from only SmartFarm data}. Theoretically this should provide a poor estimate as SmartFarm does not provide all the parameters required to estimate ET. However, it might be interesting to observe the manner in which the estimates deviate from that of CIMIS.
\end{enumerate}

\end{document}
